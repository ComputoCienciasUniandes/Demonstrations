\documentclass{beamer}
%\documentclass[handout]{beamer}
\usepackage[spanish]{babel}
\usepackage[utf8]{inputenc}
\usepackage{graphicx}
\usepackage{multimedia}
\usepackage{animate}
\usepackage{tcolorbox}
\usebackgroundtemplate{\includegraphics[width=\paperwidth]{sources/images/template_internal.jpg}}
\setbeamercolor{frametitle}{fg=white}
\usefonttheme{structuresmallcapsserif}
\setbeamertemplate{footline}[frame number]

\usepackage{default}

\newcounter{stepsBarnes}
\newcommand{\seti}{\setcounter{stepsBarnes}{\value{enumi}}}
\newcommand{\conti}{\setcounter{enumi}{\value{stepsBarnes}}}
\definecolor{green}{RGB}{0, 150, 0}

\begin{document}
{
	\usebackgroundtemplate{\includegraphics[width=\paperwidth]{sources/images/template_main.jpg}}
	\begin{frame}
		\centering
		\color{white}
		\textsc{\LARGE Dinámica de galaxias, una simulación con $N\log N$ iteraciones.}
		\\
		\vspace{5cm}
		\raggedleft Juan Barbosa
	\end{frame}
}

\begin{frame}{Introducci\'on}
	\centering
	\begin{columns}
		\begin{column}{.5\textwidth}
			\includegraphics[height=0.8\textheight]{sources/images/barnes1986.pdf}
		\end{column}
		\begin{column}{.5\textwidth}
			\centering
			\begin{tabular}{cc}
				\includegraphics[height=0.3\textheight]{sources/images/barnes.jpg} & Josh Barnes\\
				\includegraphics[height=0.3\textheight]{sources/images/hut.jpg}
				& Piet Hut \\
			\end{tabular}
		\end{column}
	\end{columns}
	
\end{frame}

\begin{frame}{Introducci\'on}
	\begin{columns}
		\begin{column}{.5\textwidth}
			\begin{tabular}{c}
				\includegraphics[width=\textwidth]{sources/images/DEC-VAX-11-780} \\
				VAX 11/780
			\end{tabular}
			\begin{itemize}
				\item CPU 5 MHz, 2 kB cache.
				\item Memoria 8 MB.
			\end{itemize}
		\end{column}
		\begin{column}{.5\textwidth}
			Barnes y Hut realizaron una simulaci\'on con su m\'etodo.
			\begin{itemize}
				\item<2-> 4096 cuerpos.
				\item<3-> 10 horas de CPU.
			\end{itemize}
		\end{column}
	\end{columns}
\end{frame}
\begin{frame}{Funcionamiento}
	El algoritmo propuesto por Barnes y Hut consta de dos pasos fundamentales, la divisi\'on recursiva del espacio y aproximaciones en la fuerza sobre un cuerpo. \pause
	\begin{enumerate}
		\item Divisi\'on jer\'arquica del espacio. $\rightarrow$ Construcci\'on de un \'arbol.\pause
		\seti
	\end{enumerate}
	\begin{columns}
		\begin{column}{0.5\linewidth}
			\centering
			\includegraphics[width=\linewidth]{sources/images/quadtree.png}\\
			\'Arbol dos dimensional.\pause
		\end{column}
		\begin{column}{0.5\linewidth}
			\centering
			\includegraphics[width=\linewidth]{sources/images/octree.png}\\
			\'Arbol tridimensional.
		\end{column}
	\end{columns}
\end{frame}
\begin{frame}{Funcionamiento}
	\begin{enumerate}
		\conti
		\item Fuerza sobre un cuerpo.
	\end{enumerate}
	\small
	El n\'umero de iteraciones se reduce al considerar centros de masa, y un coeficiente de precisi\'on $\tau$. \pause
	\begin{itemize}
		\item Cada caja tiene una longitud espec\'ifica \pause
		\item Un centro de masa \pause
		\item Y la masa contenida \pause
	\end{itemize}
	\centering
	\includegraphics[width=0.5\linewidth]{sources/images/quadtree.png}
\end{frame}
\begin{frame}{Funcionamiento}
	Pasos en el c\'alculo de la fuerza sobre un cuerpo. El proceso empieza desde el nodo superior.
	\begin{enumerate}
		\item Se calcula la distancia entre el centro de masa del nodo y el cuerpo.
		\item Si la distancia es lo suficientemente grande, los cuerpos contenidos en el nodo se toman como un solo cuerpo.
		\item De lo contrario el proceso se repite con los nodos inferiores.
	\end{enumerate}
\end{frame}
\begin{frame}{Funcionamiento}
	Para determinar si un nodo se encuentra lo suficientemente lejos de un cuerpo.
	\begin{equation}
		\tau' = \dfrac{s}{d}
	\end{equation}
	
	Donde $s$ corresponde con la longitud de la caja, y $d$ con la distancia.
	
	Se define un coeficiente de precisi\'on $\tau$ para la simulaci\'on tal que:
	\begin{itemize}
		\item Si $\tau' > \tau$, el nodo es cercano.
		\item Si $\tau' \leq \tau$, el nodo es distante. 
	\end{itemize}
\end{frame}
\begin{frame}{Funcionamiento}
	\begin{columns}
		\begin{column}{0.6\textwidth}
			\centering
			\includegraphics[width=\linewidth]{sources/images/force.png}\\
			\begin{itemize}
				\item Coeficiente de precisi\'on.
			\end{itemize}
			$\tau = \dfrac{size}{distance} = 0.5$
			\pause
		\vspace{1cm}
		
		\end{column}
		\begin{column}{0.4\textwidth}
			\begin{enumerate}
				\footnotesize
				\item {\color{orange} Nodo principal} \pause
				\begin{equation*}
					\dfrac{s}{d} = \dfrac{100}{55.7} \approx 1.8 > \tau 
				\end{equation*}
				\pause
				\item {\color{green} Primer nodo} \pause
				\item {\color{green} Segundo nodo} \pause
				\begin{equation*}
					\dfrac{s}{d} = \dfrac{50}{77.4} \approx 0.6 > \tau
				\end{equation*}
				\pause
				\item {\color{blue} Segundo nodo} \pause
				\boxed{\dfrac{s}{d} = \dfrac{25}{77.4} \approx 0.3 < \tau}
				\pause
				\item {\color{green} Cuarto nodo} \pause
				\boxed{\text{Nodo externo, contribuye}}
			\end{enumerate}
		\end{column}
	\end{columns}
\end{frame}
\begin{frame}{Funcionamiento}
	La construcci\'on del \'arbol se realiza para cada instante de tiempo.
	\centering
	\movie[height = 0.55\textwidth, width = 0.8\textwidth, poster, showcontrols]{}{sources/animations/boxes_points.mp4}
	\\
	Observaci\'on
\end{frame}
\begin{frame}{Funcionamiento}
	La construcci\'on del \'arbol se realiza para cada instante de tiempo.
	\centering
	\movie[height = 0.55\textwidth, width = 0.8\textwidth,poster, showcontrols]{}{sources/animations/boxes.mp4}
	\\
	Todas las cajas
\end{frame}
\begin{frame}{Funcionamiento}
	La construcci\'on del \'arbol se realiza para cada instante de tiempo.
	\centering
	\movie[height = 0.55\textwidth, width = 0.8\textwidth, poster, showcontrols]{}{sources/animations/boxes_child_only.mp4}
	\\
	Cajas con una part\'icula
\end{frame}
\begin{frame}{Construcci\'on de una simulaci\'on}
	\begin{enumerate}
		\item Descripci\'on del sistema. \pause
		\item Condiciones iniciales. \pause
		\item Soluci\'on de las ecuaciones. \pause
		\item Visualizaci\'on.
	\end{enumerate}
\end{frame}
\begin{frame}{Descripci\'on del sistema}
	Usando la ley de gravitaci\'on universal:
	\begin{equation}
		\vec{F_i} = m_i\vec{a}_i = - \sum\limits_{j\neq i}^N G\dfrac{m_im_j}{|\vec{r}_i - \vec{r}_j|^3}\left(\vec{r}_i - \vec{r}_j\right)
	\end{equation}\pause
	es posible obtener las ecuaciones que describen la din\'amica del sistema.\pause
	\begin{equation}
		\vec{\ddot{r}}_i = - \sum\limits_{j\neq i}^N G\dfrac{m_j}{\left(|\vec{r}_i - \vec{r}_j|^2 + \epsilon^2\right)^{3/2}}\left(\vec{r}_i - \vec{r}_j\right)
	\end{equation}
	
	donde $\epsilon$ corresponde con el \textit{softening length}.
\end{frame}
\begin{frame}{Condiciones iniciales}
	Suponiendo \'orbitas circulares y teniendo en cuenta la masa encerrada en las \'orbitas de menor tama\~no:
	\begin{equation}
		v \approx \sqrt{\dfrac{GM(r)}{r}}
	\end{equation}\pause
	
	En coordenadas polares:
	\begin{equation}
		\begin{matrix}
			x = r\cos(\theta) \qquad \longrightarrow \qquad \dot{x} = -r\sin(\theta) = -y \\
			y = r\sin(\theta) \qquad \longrightarrow \qquad \dot{y} = r\cos(\theta) = x\\
			z = z \longrightarrow \dot{z} = \dot{z} \qquad??? \\
		\end{matrix}
	\end{equation}
\end{frame}
\begin{frame}{Condiciones iniciales}
	\begin{columns}
		\begin{column}{0.5\textwidth}
			\includegraphics[height=0.35\textheight]{sources/images/galaxy_shape.pdf}\\\pause
			\includegraphics[height=0.35\textheight]{sources/images/rotation_curve.pdf}\pause
		\end{column}
		\begin{column}{0.65\textwidth}
			\includegraphics[height=0.75\textheight]{sources/images/galaxy_distribution.pdf}\\
		\end{column}
	\end{columns}
\end{frame}
\begin{frame}{Condiciones iniciales}
	\begin{columns}
		\begin{column}{0.5\textwidth}
			\includegraphics[height=0.35\textheight]{sources/images/galaxy_shape.pdf}\\
			\includegraphics[height=0.35\textheight]{sources/images/rotation_curve.pdf}
		\end{column}
		\begin{column}{0.65\textwidth}
			\footnotesize
			\begin{itemize}
				\item $G$ = 44.97 (10$^7$ M$_{\odot}^{-1}$kpc$^3$Gy$^{-2}$)
				\item Tama\~no $\approx$ 80 (kpc)
				\item $m$ = 2.44 (10$^7$M$_{\odot}$)
				\item $N$ = 4096
				\item $\epsilon$ = 0.055
			\end{itemize}
		\end{column}
	\end{columns}
\end{frame}
\begin{frame}{Soluci\'on de las ecuaciones}
	{\scshape Leapfrog}
	La soluci\'on num\'erica a las ecuaciones diferenciales se obtiene usando el m\'etodo de \textit{Leapfrog}, el cual avanza asincr\'onicamente en la posici\'on y la velocidad.
	\begin{equation}
		\begin{matrix}
			v_{i+1/2} = v_i + a_i\dfrac{\Delta t}{2}\\
			x_{i+1} = x_i + v_{i+1/2}\Delta t \\
			v_{i+1} = v_{i+1/2}+a_{i+1}\dfrac{\Delta t}{2}
		\end{matrix}
	\end{equation}
\end{frame}

\begin{frame}{Visualizaci\'on}
	La visualizaci\'on de los resultados se realiza usando animaciones de las posiciones en funci\'on del tiempo. Tambi\'en se obtiene informaci\'on de las curvas de rotaci\'on, distribuci\'on de masa y velocidades.
\end{frame}
\begin{frame}{Implementaci\'on}
	Binding de \texttt{bruteforce.c} para Python.
	\begin{tcolorbox}[colback=green!5,colframe=green!40!black,title=C Programming Language]
		\begin{itemize}
			\item \texttt{init.c}: configura las variables globales de la simulaci\'on $(N, m, G, \epsilon, \tau)$, los arrays de posiciones y velocidades.
			\item \texttt{box.c}: contiene las funciones propias del \'arbol y sus cajas.
			\item \texttt{bruteforce.c}: resuelve las ecuaciones diferenciales, y genera archivos de datos.
		\end{itemize}
	\end{tcolorbox}
\end{frame}
\begin{frame}{Implementaci\'on}
	\begin{tcolorbox}[colback=blue!5,colframe=blue!40!black,title=Python]
		\begin{itemize}
			\item \texttt{core.py}: contiene la clase \texttt{Galaxy} y \texttt{Simulation}, las cuales generan condiciones iniciales y realizan la interfaz con C.
		\end{itemize}
	\end{tcolorbox}
\end{frame}
\begin{frame}{Ejecuci\'on de la simulaci\'on}
	\begin{columns}
		\begin{column}{0.75\textwidth}
			\includegraphics[height=0.5\textheight]{sources/images/code.png}
		\end{column}
		\begin{column}{0.25\textwidth}
			Python script
		\end{column}
	\end{columns}
	\footnotesize
	\begin{enumerate}
		\item \texttt{galaxy1, galaxy2, system, speeds = example(N, M, G)} \pause
		\item \texttt{sim = Simulation(M, G, system, speeds, epsilon, tolerance = 1.0, threads = -1)} \pause
		\item \texttt{sim.start(0.0, 1.0, 0.01)}
	\end{enumerate}
\end{frame}
\begin{frame}{Ejecuci\'on de la simulaci\'on}
	\begin{columns}
		\begin{column}{0.75\textwidth}
			\includegraphics[height=0.5\textheight]{sources/images/screen.png}
		\end{column}
		\begin{column}{0.25\textwidth}
			Python Notebook
		\end{column}
	\end{columns}
	Disponible en:
	\tiny
	\url{https://github.com/ComputoCienciasUniandes/Demonstrations/tree/master/GalaxyDynamics}
	\normalsize
\end{frame}
\begin{frame}{Resultados}
	\centering
	\movie[height = 0.55\textwidth, width = 0.8\textwidth, poster, showcontrols]{}{sources/animations/exact.mp4}
	\\
\end{frame}
\begin{frame}{Resultados}
	\centering
	\movie[height = 0.55\textwidth, width = 0.8\textwidth, poster, showcontrols]{}{sources/animations/approx.mp4}
	\\
\end{frame}
\begin{frame}{Resultados}
	Efecto del coeficiente de precisi\'on en el tiempo de c\'omputo.
	\begin{columns}
		\begin{column}{0.5\textwidth}
			\includegraphics[width=\linewidth]{sources/images/Tolerance_results.pdf}
		\end{column}
		\begin{column}{0.5\textwidth}
			\includegraphics[width=\linewidth]{sources/images/Particles_results.pdf}
		\end{column}
	\end{columns}
\end{frame}

\begin{frame}{Resultados}
	Conservaci\'on de la energ\'ia para $\tau = 1.0$.
	\centering
	\includegraphics[width=0.8\linewidth]{sources/images/energy.pdf}
\end{frame}
\begin{frame}{Conclusiones}
	\begin{itemize}
		\item El m\'etodo de Barnes y Hut constituye una buena aproximaci\'on para las interacciones gravitacionales de $N$ cuerpos.
		
		\item El \'orden del algoritmo es $\mathcal{O}(N\log N)$.
		
		\item El costo computacional disminuye exponencialmente con el valor de $\tau$.
	\end{itemize}
\end{frame}
\end{document}